%!TEX root = ../main.tex

\chapter{Richiami di Matematica} % (fold)
\label{cha:matematica}

\section{Regole di derivazione} % (fold)
\label{sec:regole_di_derivazione}

Siano $f(x)$ e $g(x)$ funzioni reali di variabile reale $x$ derivabili, e sia $\mathrm{D}$ l'operazione di derivazione rispetto a $x$:
\begin{align*}
    \mathrm{D}[f(x)]=f'(x) \qquad \mathrm{D}[g(x)]=g'(x)
\end{align*}
\begin{itemize}
    \item \textbf{Regola della somma}:
    \begin{align*}
        \mathrm{D}[\alpha f(x)+ \beta g(x)] = \alpha f'(x) + \beta g'(x) \qquad \alpha, \beta \in \mathbb{R}
    \end{align*}
    \item \textbf{Regola del prodotto}:
    \begin{align*}
        \mathrm{D} [ {f(x) \cdot g(x)}] = f'(x) \cdot g(x) + f(x) \cdot g'(x) 
    \end{align*}
    \item \textbf{Regola del quoziente}:
    \begin{align*}
        \mathrm{D}\! \left[ {f(x) \over g(x)} \right] = { f'(x)  \cdot g(x) - f(x) \cdot g'(x) \over g(x)^2}
    \end{align*}
    \item \textbf{Regola della funzione reciproca}:
    \begin{align*}
        \mathrm{D}\! \left[ {1 \over f(x)} \right] = -{f'(x) \over f(x)^2} 
    \end{align*}
    \item \textbf{Regola della funzione inversa}:
    \begin{align*}
        &\mathrm{D}[f^{-1}(y)]  =  {1 \over f'(x)} \\
        &\text{con:} \\
        &y = {f(x)} \qquad x = {f^{-1}(y)}
    \end{align*}
    \item \textbf{Regola della catena}
    \begin{align*}
        \mathrm{D} \left[ f \left( g(x) \right) \right] = f' \left( g(x) \right) \cdot g'(x) 
    \end{align*}
\end{itemize}

\section{Integrali}

\begin{thm}[Teorema Fondamentale del Calcolo Integrale (prima parte)]
	
Sia $f\colon [a,b]\to\mathbb{R}$ una funzione integrabile. Si definisce funzione integrale di $f$ la funzione $F$ tale che:
\begin{align*}
	F(x)=\int_a^x f(t)dt \qquad a \le x \le b
\end{align*}
Se $f$ è limitata, allora $F$ è una funzione continua in $[a,b]$. 

Se inoltre $f$ è una funzione continua in $(a,b)$, allora $F$ è funzione differenziabile in tutti i punti in cui $f$ è continua e si ha:
\begin{align*}
	F^\prime(x)=f(x)
\end{align*}

cioè $F$ risulta essere una primitiva di $f$.
\end{thm}

% section derivate (end)
% chapter matematica (end)
